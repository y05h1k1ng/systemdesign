% Created 2019-11-13 水 19:47
% Intended LaTeX compiler: pdflatex
\documentclass[11pt]{article}
\usepackage[utf8]{inputenc}
\usepackage[T1]{fontenc}
\usepackage{graphicx}
\usepackage{grffile}
\usepackage{longtable}
\usepackage{wrapfig}
\usepackage{rotating}
\usepackage[normalem]{ulem}
\usepackage{amsmath}
\usepackage{textcomp}
\usepackage{amssymb}
\usepackage{capt-of}
\usepackage{hyperref}
\author{Yoshiki KITAMURA}
\date{\today}
\title{設計仕様書(ソフトウェア)}
\hypersetup{
 pdfauthor={Yoshiki KITAMURA},
 pdftitle={設計仕様書(ソフトウェア)},
 pdfkeywords={},
 pdfsubject={},
 pdfcreator={Emacs 26.3 (Org mode 9.1.9)}, 
 pdflang={English}}
\begin{document}

\maketitle
\tableofcontents


\section{api}
\label{sec:orgef6202d}
自作apiを用いる。

理由
\begin{itemize}
\item 本物のapiを使うとそもそも災害が起きないので、テストがしやすい。
\item 存在するapiは有料であったりする。
\end{itemize}

\subsection{json format}
\label{sec:org0569408}
\begin{verbatim}
{
  "city": string,
  "earthquake": {
    "level": int,
    "hoge" : string,
  },
  "rain": {
    "level": int,
    "hoge" : string,
  },
}
\end{verbatim}

\section{web service}
\label{sec:org1e95792}
保管しているものを管理する。
flask製

\begin{center}
\begin{tabular}{ll}
もの & 消費期限\\
\hline
ビスケット & 2019/11/11\\
カップラーメン & 2020/01/24\\
なべ & null\\
\end{tabular}
\end{center}

疑問点
\begin{itemize}
\item 食べ物以外も表示させておく?
\end{itemize}
\end{document}
